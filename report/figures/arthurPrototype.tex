\begin{figure}
\begin{center}
\begin{verbatim}
public void timingEvent(Animator source, double fraction){
    boolean done = false;
    if(anArr.getToDo().size() > 0){
        Step[] steps = new Step[anArr.getToDo().peek().length];
        System.arraycopy(anArr.getToDo().peek(), 0, steps, 0, steps.length);

        for(int i = 0; i < steps.length; i++){                    
            Rectangle rect = rect_list[steps[i].getIndex()].getRec();
            double x = steps[i].getX();
            double y = steps[i].getY();
            double stepX = rect.getX();
            double stepY = rect.getY();
            if(y != stepY){
                if(Math.abs(x - stepX) < 50){
                    if(y > stepY){
                        rect_list[steps[i].getIndex()].getRec().x += 1;
                    }
                    else{
                        rect_list[steps[i].getIndex()].getRec().x -= 1;
                    }
                }
                else{
                    if(y > stepY){
                        rect_list[steps[i].getIndex()].getRec().y += 1;
                    }
                    else{
                        rect_list[steps[i].getIndex()].getRec().y -= 1;
                    }
                }
            }
            else{
                if(x > stepX){
                    rect_list[steps[i].getIndex()].getRec().x += 1;
                }
                else{
                    rect_list[steps[i].getIndex()].getRec().x -= 1;
                }
            }
            if(y == stepY && x == stepX){
                done = true;
                anArr.getToDo().poll();
                anArr.getDone().add(steps);
                break;
            }
        }
        repaint();              

    }
\end{verbatim}
\end{center}
\caption{Arthur example code making an animated change in coordinates}
\label{fig:arthurPrototype}
\end{figure}