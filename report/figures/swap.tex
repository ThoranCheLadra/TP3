\begin{figure}
\begin{center}
\begin{verbatim}
Step s = new Step(); // create a new step
// Create animation for showing information on what's happening
String information;
if (info == "") {
    information = "Swapping index " + a + " (" + rect_list.get(a).getLabel() + 
        ") " + " with index " + b + " (" + rect_list.get(b).getLabel() + ")"; // create the string
} else {
    information = info;
}
s.setInfo(information); // add it to the step
// Change information
s.addAnimator(new Animator.Builder().setDuration(1, TimeUnit.MILLISECONDS)
    .build(), s.getFirstAnimator()); // create an animator for this, which 
                                     //we could use to trigger this change
s.getLastAnimator().addTarget(new ChangeLabel(this, information, this));
// create a timing target to change the label when the animation starts	
// Create the Animators and PropertySetters (what should the animation change 
// the property from what to what) for swapping rects
s.addAnimator(new Animator.Builder().setDuration(time, TimeUnit.MILLISECONDS)
	.build(), s.getFirstAnimator()); // we want this Animator to start the same
// time the first Animator in the step does
s.getLastAnimator().addTarget(PropertySetter.getTarget(rect_list.get(a), 
	"currentY", rect_list.get(a).getRec().y, rect_list.get(a).getRec().y+rectSpace()));
s.addAnimator(new Animator.Builder().setDuration(time, TimeUnit.MILLISECONDS)
	.build(), nextBtn); // we want this Animator to start straight after the 
	// last Animator finishes. This is how it's done just now
s.getLastAnimator().addTarget(PropertySetter.getTarget(rect_list.get(a), 
"currentX", rect_list.get(a).getRec().x, rect_list.get(b).getRec().x)); 
s.addAnimator(new Animator.Builder().setDuration(time, TimeUnit.MILLISECONDS).build(), nextBtn);
s.getLastAnimator().addTarget(PropertySetter.getTarget(rect_list.get(a),
 "currentY", rect_list.get(a).getRec().y+rectSpace(), rect_list.get(a).getRec().y));		
//rect 2
s.addAnimator(new Animator.Builder().setDuration(time,
 TimeUnit.MILLISECONDS).build(), s.getFirstAnimator());															// we want this Animator to start the same time the first Animator in the step does
s.getLastAnimator().addTarget(PropertySetter.getTarget(rect_list.get(b),
 "currentY", rect_list.get(b).getRec().y, rect_list.get(b).getRec().y-rectSpace()));
s.addAnimator(new Animator.Builder().setDuration(time, TimeUnit.MILLISECONDS).build(), nextBtn);
s.getLastAnimator().addTarget(PropertySetter.getTarget(rect_list.get(b),
 "currentX", rect_list.get(b).getRec().x, rect_list.get(a).getRec().x));
s.addAnimator(new Animator.Builder().setDuration(time, TimeUnit.MILLISECONDS).build(), nextBtn);
s.getLastAnimator().addTarget(PropertySetter.getTarget(rect_list.get(b),
 "currentY", rect_list.get(b).getRec().y-rectSpace(), rect_list.get(b).getRec().y));
// Trigger for a continuous animation
s.addAnimator(new Animator.Builder().setDuration(1, TimeUnit.MILLISECONDS).build(), nextBtn);
s.getLastAnimator().addTarget(new ContinuousAnimation(currentStep+1, this));
// Log the changes in the Step
s.addChanges(new Change("swap", rect_list.get(a),
 rect_list.get(a).getRec().x, rect_list.get(a).getRec().y));
s.addChanges(new Change("swap", rect_list.get(b),
 rect_list.get(b).getRec().x, rect_list.get(b).getRec().y));
\end{verbatim}
\end{center}
\caption{Sample code from the AnimatedArray swap function}
\label{fig:swap}
\end{figure}